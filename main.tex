\documentclass[11p]{article}
% Packages
\usepackage{amsmath}
\usepackage{graphicx}
\usepackage{fancyheadings}
\usepackage[swedish]{babel}
\usepackage[
    backend=biber,
    style=authoryear-ibid,
    sorting=ynt
]{biblatex}
\usepackage[utf8]{inputenc}
\usepackage[T1]{fontenc}
\usepackage{textcomp}
%Källor
\addbibresource{references.bib}
\graphicspath{ {./images/} }

% Lite variabler
\def\email{neobackstrom.desiatnik@elev.ga.ntig.se}
\def\foottitle{Bortom utseendet}
\def\name{Neo Desiatnik, TE21}

\title{Bortom utseendet \\ \small Påverkar sällsynthet hur bra något ser ut?}
\author{\name}
\date{\today}

\begin{document}

% fixar sidfot
\lfoot{\footnotesize{\name \\ \email}}
\rfoot{\footnotesize{\today}}
\lhead{\sc\footnotesize\foottitle}
\rhead{\nouppercase{\sc\footnotesize\leftmark}}
\pagestyle{fancy}
\renewcommand{\headrulewidth}{0.2pt}
\renewcommand{\footrulewidth}{0.2pt}

% i Sverige har vi normalt inget indrag vid nytt stycke
\setlength{\parindent}{0pt}
% men däremot lite mellanrum
\setlength{\parskip}{10pt}

\maketitle
\section{Vad är Anime Adventures?} Anime Adventures är ett spel som finns på Roblox. Spelet är i kategorin ``Tower Defense`` och det går ut på att man ska placera saker som skyddar din bas från fiender. Det tilliknar den klassiska spel serien ``Bloons tower defense``. Man får units genom att spendera spelpengar.
\subsection{Units} I spelet som jag undersöker så finns det karaktärer som heter ``units``. Man använder dem genom att placera dom så de kan förstöra fiender så de inte tar sig till din bas och förstör den.
\subsection{Skins} Det finns väldigt få skins i spelet och ännu färre skins har cosmetics. Dessa skins kan man lägga på olika units och de ändrar unitens utseende.
\subsection{Cosmetics} Några units och skins i spelet har cosmetics, det är accessoarer som sätts på din spelkaraktär för att göra den mer unik.
\subsection{Rarities} Rarity är ett värde som indikerar hur sällsynt en unit är. Om en unit har hög rarity så indikerar det ofta att den har hög value eftersom det var svårt att få tag på den.
\subsection{Shinies} När man får en unit finns det chans att få den shiny. De flesta shinies har andra cosmetics än den vanliga uniten. Om en unit är shiny så ändrar den inte hur bra den är, utan endast utseende.
\subsection{Avatar} Din spelkaraktär som är universiell för hela roblox. Man kan spendera pengar för att ändra på utseendet och det resulterar i att alla avatars är annourlunda.
\subsection{Discord} En chatting app som har stora chattrum som personer kan delta i. Dessa chattrum kan användas för allt möjligt men det som jag refererar till när jag säger discord server är Anime Adventures officiella discord server.
\subsection{Communities} Personer som spelar ett spel och deltar i diskussioner om spelet och jobbar tillsammans (eller mot varandra) för att komma fram till strategier för spelet.
\subsection{Values} Alla units i spelet har en value som baseras på hur svårt det är att få uniten och hur bra den är, i kobination till hur många som vill ha den. Value kan variera från 5 value till 18,500. Det finns två units som har värdet ``Owner's choice`` vilket är när en unit är så sällsynt att bara dom som har uniten kan bestämma value själv.
\subsection{Trading} När två spelare är i samma server så kan man skicka trade requests, om man skickar en trade request så kan den andra personen välja att acceptera eller inte. Om man accepterar så kan båda spelarna lägga in units eller skins av varierande value för att få den andra personens units eller skins. Om båda accepterar en trade så kommer sakerna som spelarna har lagt in bytas.
\clearpage

\end{document}